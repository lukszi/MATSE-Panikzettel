\documentclass[12pt,a4paper]{article}
\usepackage{german,a4}
\usepackage{bookmark}
%\usepackage[latin1]{inputenc}
\usepackage[utf8]{inputenc}
\usepackage{amssymb}
\usepackage{amsmath}
\usepackage{multirow}
\usepackage{rotating}
\usepackage{pstricks}
\usepackage{svg}
\usepackage{graphicx}
\graphicspath{ {./images/} }

\renewcommand{\labelenumi}{\arabic{enumi}.)}
\renewcommand{\familydefault}{\sfdefault}
\headsep 0cm
\headheight 0cm
\textheight 25cm
\textwidth 16cm
\oddsidemargin 0cm
\parindent 0cm


\begin{document}

\textbf{Panikzettel} \\ 
\textbf{Analysis 1, WS 2017/2018} \\
{\scriptsize (Lukas Szimtenings)}
\hrule 

\bigskip

\tableofcontents
\pagebreak
\section{Grundlagen}

\subsection{Eigenschaften von Mengen}

\subsubsection{basics}
Sei $G$ die Menge eines geordneten Körpers, $F$ Die Menge aller oberen Schranken
\begin{itemize}
\item $a$ ist Obere Schranke: $a\geq x\quad \forall x \in G$ (Analog für Untere Schranke)
\item $a$ ist Supremum: $a\leq x \quad \forall x \in F$ (Analog für Infimum)
\item $a$ ist Maximum: $a$ ist Supremum von $G$ und $a \in G$ (Analog für Minimum)
\end{itemize}


\subsubsection{advanced}
\textbf{$\epsilon$-Umgebung}\\
\begin{figure}[htbp]
  \centering
  \includesvg{Epsilon_Umgebung}
  \caption{Epsilon Umgebung}
\end{figure}
wenn $a\in A$ ist die Epsilonumgebung definiert als:\\
\[U_{\epsilon}(a)=\{ x\in A | x>a-\epsilon \wedge x<a+\epsilon\}\]


\textbf{innerer Punkt}: 
$a_0\in A$ ist innerer Punkt wenn gilt: $\exists \epsilon : U_{\epsilon}(a_0)\subset A$, es also mindestens eine Epsilonumgebung gibt die komplett in der Menge A enthalten ist\\
Offene Menge: Alle Punkte der Menge sind innere Punkte\\

\textbf{Häufungspunkt}: 
a ist Häufungspunkt wenn gilt: Für jedes noch so kleine $\epsilon$ existiert eine $\epsilon$-Umgebung die mindestens einen Punkt aus A enthält der nicht a ist.\\
$\forall \epsilon>0:\quad \exists x \in A, x \in U_{\epsilon}(a) $\\

\textbf{Abgeschlossenheit}: 
A ist abgeschlossen: jeder Häufungspunkt liegt in A\\
Folgende Mengen sind abgeschlossen:
\begin{itemize}
	\item jede endliche Menge
	\item $\mathbb{N}$
	\item $[a,b]$ 
\end{itemize}

\pagebreak
\section{Folgen und Reihen}
\subsection{Rekursion}
\subsubsection{Wichtige rekursive Folgen}

\textbf{arithmetische Folge:}
\begin{itemize}
\item definiert als: $a_n = a_{n-1} + c $
\item Bildungsgesetz: $a_n = c\cdot n+a_0$
\item Konstant mit: $\Delta a := a_n - a_{n-1} = c$
\end{itemize}

\textbf{geometrische Folge:}
\begin{itemize}
\item definiert als: $a_n = a_{n-1}\cdot c$
\item Bildungsgesetz: $a_n = a_0\cdot c^n$
\item Konstant mit: $\frac{a_n}{a_{n-1}}=c$
\item Mit $c>1$ ist $a_n$ wachsend
\item Mit $c<1$ ist $a_n$ fallend, mit $a_0>0$ aber immer positiv
\end{itemize}

\subsubsection{Arithmetisch oder geometrisch :}
wenn die Differenz $a_n-a_{n-1}$ Konstant ist, arithmetisch\\
wenn der Quotient $\frac{a_n}{a_{n-1}}$ Konstant ist, geometrisch\\

\subsection{Summen(Reihen):}
\subsubsection{Wichtige Rechenregeln:}
\[\sum_{k=m}^nc=c\cdot(n-m+1)\]
\[\sum_{k=m}^na_k=\sum_{k=m+l}^{n+l}a_{k-l}\]

\subsubsection{Wichtige Summen}
\textbf{arithmetische Summe}
\[\sum_{k=1}^nk=\frac{n\cdot(n+1)}{2}\]
\textbf{geometrische Summe}
\[\sum_{k=0}^nc^k=\frac{1-c^{n+1}}{1-c}\]
mit $c=1$
\[\sum_{k=0}^n1^k=n+1\]

\subsection{Binomialkoeffizient}
\subsubsection{Definition}
\[\binom{n}{k}=\begin{cases}
					0 &\quad k>n\\
					\frac{n!}{(n-k)!\cdot k!} &\quad k\leq n\\
				\end{cases}\]

\subsubsection{Rechenregeln}
Im folgenden gelte $k\leq n$:\\

\begin{itemize}
\item $\binom{n}{0}=1$
\item $\binom{n}{n}=1$
\item $\binom{n}{1}=n$
\item $\binom{n}{k}=\binom{n-1}{k}+\binom{n-1}{k-1}$
\end{itemize}

\subsubsection{Binomischer Lehrsatz}
$(a+b)^n=\sum_{k=0}^n \binom{n}{k}a^k\cdot b^{n-k}$

\pagebreak
\section{Konvergenz von Folgen}
\subsection{Grenzwert}
$a$ ist Grenzwert von $a_n$ wenn gilt:\\
Für jedes noch so kleine $\epsilon>0$ gibt es ein $n_0(\epsilon)$ ab dem alle $a_n$ näher als $\epsilon$ an $a$ liegen.\\
In der Praxis $|a_n-a|<\epsilon$ mit Abschätzungen auf $n>\epsilon...$ umformen\\\\
Eine Folge die einen Grenzwert besitzt heißt Konvergent\\
Eine Folge die den Grenzwert 0 besitzt heißt Nullfolge\\
\textbf{jede beschränkte monotone Folge ist konvergent}

\subsection{Sandwich-Lemma}
gegeben seien $a_n$, $b_n$ und $c_n$ mit $a_n = b$, $c_n = b$\\
existiert ein $n_0$ so dass, $\forall n>n_0:\quad a_n<b_n<c_n$, dann gilt auch $b_n=b$

\subsection{Rechenregeln}
\begin{itemize}
	\item $\lim\limits_{n\rightarrow \infty} (a_n+b_n)=\lim\limits_{n\rightarrow \infty}a_n+\lim\limits_{n\rightarrow \infty}b_n$
	\item $\lim\limits_{n\rightarrow \infty} (c\cdot a_n)=c\cdot\lim\limits_{n\rightarrow \infty}a_n$
	\item $\lim\limits_{n\rightarrow \infty} (a_n\cdot b_n)=\lim\limits_{n\rightarrow \infty}a_n\cdot\lim\limits_{n\rightarrow \infty}b_n$
	\item $\lim\limits_{n\rightarrow \infty} (\frac{a_n}{b_n})=\frac{\lim\limits_{n\rightarrow \infty}a_n}{\lim\limits_{n\rightarrow \infty}b_n}$
\end{itemize}

\subsection{Wichtige Grenzwerte}
\begin{itemize}
	\item $\lim\limits_{n\rightarrow \infty} \frac{1}{n^a}=0\quad$ für $a>0$
	\item $\lim\limits_{n\rightarrow \infty} \sqrt[n]{a}=1\quad$ für $a>0$
	\item $\lim\limits_{n\rightarrow \infty} q^n=0\quad$ für $|q|<1$
	\item $\lim\limits_{n\rightarrow \infty} n^k\cdot q^n=0\quad$ für $|q|<1, k\in\mathbb{N}, q\in\mathbb{R}$
	\item $\lim\limits_{n\rightarrow \infty} \sqrt[n]{n}=1$
	\item $\lim\limits_{n\rightarrow \infty} \frac{n!}{n^n}=0$
\end{itemize}


\subsection{Rekursive Folgen}
\begin{enumerate}
	\item Monotonie prüfen durch Werte ausrechnen oder Mathematikerbrille
	\item Monotonie per Induktion zeigen
	\item Beschränktheit zeigen 
	\item \begin{enumerate}
		\item geeignete Schranke wählen
		\item per Induktion zeigen
	\end{enumerate}
	\item so tun als kenne man den Grenzwert a und mit lim$a_{n+1}$ gleichsetzen
	\item $a_{n+1}$ auflösen und dann nach a umformen
\end{enumerate}


\subsection{Cauchy-Konvergenz}
\begin{figure}[htbp]
	\centering
	\includesvg{Cauchy_Folge}
	\caption{Cauchy-Folge}
\end{figure}
Alle Werte einer Cauchy Folge liegen ab einem bestimmten n beliebig nah aneinander, sprich die Differenz ist kleiner als ein beliebiges $\epsilon>0$.\\
\[\forall \epsilon > 0 \exists n_0(\epsilon) \text{ mit } |a_n-a_m|<\epsilon\quad\forall n> m\geq n_0\]
jede Cauchy-konvergente Folge ist Konvergent und umgekehrt
\paragraph{Schema F}
\begin{itemize}
	\item Den Betrag durch abschätzen entfernen
	\item Am besten n durch abschätzen entfernen
	\item Am ende sollte $\epsilon>\text{ irgendwas}$ rauskommen.
	\item Nach m oder n umformen
	\item $n_0$ wählen, indem man den Term mit $\epsilon$ nach unten rundet und eins addiert.
\end{itemize}
\pagebreak

\section{Konvergenz von Reihen}
\subsection{Die Folge der Partialsummen}
\textbf{Partialsumme}
Eine Partialsumme ist ein Glied einer Folge die entsteht wenn man die Elemente einer anderen Folge aufsummiert\\
Formal: \[S_n :=\sum_{k=1}^{n}a_k\]
$S_n$ heißt dann die n-te Partialsumme\\
Der Limes der Partialsummen ist eine Reihe\\
\subsection{Wichtige Reihen}
\begin{itemize}
	\item Geometrische Reihe: $\sum_{k=0}^\infty c^k=\frac{1}{1-c}\quad\text{(für }c<1$), divergiert mit $|c|\geq 1$
	\item Harmonische Reihe: $\sum_{k=0}^\infty \frac{1}{k}=\infty$ divergiert
	\item Alternierende Reihe: $\sum_{k=0}^\infty (-1)^{n-1}\cdot a_n$ Konvergenz oder Divergenz mittels Leibniz-Kriterium
\end{itemize}

\subsection{Absolute Konvergenz}
Absolute Konvergenz ist gegeben wenn die Reihe die aus dem Betrag der Folge entsteht konvergiert.\\
\[\sum_{k=0}^n a_k \text{ konvergiert absolut wenn } \sum_{k=0}^n |a_k| \text{ konvergiert }\]
\begin{itemize}
	\item Jede geometrische Reihe konvergiert absolut
	\item absolute Konvergenz impliziert Konvergenz
	\item Konvergente Reihe die aus $a_n>=0$ besteht ist immer absolut konvergent
\end{itemize}

\subsection{Konvergenzkriterien}
\subsubsection{Majoranten und Minorantenkriterium}
\textbf{Majorantenkriterium}\\
Idee: Suche eine Folge $b_n$ die ab einem bestimmten n größer ist als $a_n$.\\
Konvergiert die Reihe die aus $b_n$ entsteht, konvergiert auch die Reihe aus $a_n$\\
$b_n$ heißt Majorante\\
\textbf{Minorantenkriterium}\\
Idee: Suche eine Folge $b_n$ die ab einem bestimmten n kleiner ist als $a_n$\\
Divergiert die Reihe die aus $b_n$ entsteht, divergiert auch die Reihe aus $a_n$\\
$b_n$ heißt Minorante\\

\subsubsection{Wurzelkriterium}
Sei $a_n\geq 0$, existiert dann $r = \lim\limits_{n\rightarrow\infty} \sqrt[n]{a_n}$, gilt:
\begin{enumerate}
	\item $r<1$		die Reihe konvergiert absolut
	\item $r>1$		die Reihe divergiert
	\item $r=1$		Keine Aussage möglich
\end{enumerate}
Das Kriterium lässt sich auch nutzen wenn $a_n<0$ ist. Statt aus $a_n$ wird r dann aus $|a_n|$ berechnet, und das Kriterium zeigt nur Konvergenz, keine absolute Konvergenz.

\subsubsection{Quotientenkriterium}
Sei $a_n> 0$, existiert dann $r = \lim\limits_{n\rightarrow\infty} \frac{a_{n+1}}{a_n}$, gilt genau wie beim Wurzelkriterium:
\begin{enumerate}
	\item $r<1$		die Reihe konvergiert absolut
	\item $r>1$		die Reihe divergiert
	\item $r=1$		Keine Aussage möglich
\end{enumerate}
Analog zum Wurzelkriterium lässt sich auch dieses Kriterium ohne Einschränkung der folge nutzen.

\subsubsection{Leibnizkriterium}
Regelt die Konvergenz der alternierenden Reihe:\\
Sei $a_n>0$, $a_n$ monoton fallend und $\lim\limits_{n\rightarrow\infty}a_n=0$, dann konvergiert
\[\sum_{n=0}^{\infty}(-1)^{n}\cdot a_n\]

\subsection{Potenzreihe}
Definiert als:
\[p(x)=\sum_{n=0}^{\infty}a_nx^n\]
konvergiert wenn:
\begin{itemize}
	\item $x=0$
	\item $\lvert x\rvert<\lim\limits_{n\rightarrow\infty}\lvert\frac{a_n}{a_{n+1}}\rvert$
	\item $\lvert x\rvert<\frac{1}{\lim\limits_{n\rightarrow\infty}\sqrt[n]{\lvert a_n \rvert}}$
\end{itemize}

\pagebreak

\section{Formelsammlung}
\[\sum_{k=1}^nk=\frac{n\cdot(n+1)}{2}\]
\[\sum_{k=0}^nc^k=\frac{1-c^{n+1}}{1-c}\quad(c\neq 1)\]
\end{document}