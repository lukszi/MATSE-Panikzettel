\documentclass[12pt,a4paper]{article}
\usepackage{german,a4}
%\usepackage[latin1]{inputenc}
\usepackage[utf8]{inputenc}
\usepackage{amssymb}
\usepackage{amsmath}
\usepackage{multirow}
\usepackage{rotating}
\usepackage{pstricks}

\renewcommand{\labelenumi}{\arabic{enumi}.)}
\renewcommand{\familydefault}{\sfdefault}
\headsep 0cm
\headheight 0cm
\textheight 25cm
\textwidth 16cm
\oddsidemargin 0cm
\parindent 0cm


\begin{document}

\textbf{Panikkzettel} \\ 
\textbf{Lineare Algebra 1, WS 2017/2018} \\
{\scriptsize (Lukas Szimtenings)}
\hrule 

\bigskip

\tableofcontents
\pagebreak
\section{analytische Geometrie}
%{\large\textbf{analytische Geometrie}}

\subsection{Skalarprodukt}
\textbf{Kriterien:}

\begin{itemize}
\item [SP1:] $\forall a,b \in \mathbb{R}^n: \langle a,b\rangle =\langle b,a\rangle $
\item [SP2:] $\forall a,b,c \in \mathbb{R}^n: \langle a,b+c\rangle =\langle a,b\rangle  + \langle a,c\rangle $
\item [SP3:] $\forall \alpha \in \mathbb{R}: \langle \alpha a,b\rangle =\alpha\langle a,b\rangle =\langle a,\alpha b\rangle $
\item [SP4:] $\forall a\in\mathbb{R}^n\setminus \{0\}:\langle a,a\rangle >0$
\end{itemize}
\textbf{Euklidsches Skalarprodukt}

Wir benutzen normalerweise das Euklidsche Skalarprodukt was wie folgt definiert ist:
\[\langle a,b\rangle:=\sum_{i=1}^n a_ib_i\]

\textbf{Geometrische Bedeutung}

\subsection{Euklidsche Norm}
\textbf{Kriterien:}

\begin{itemize}
\item [N0:] $\lVert a\rVert \in \mathbb{R}$
\item [N1:] $\lVert a\rVert \geq 0$
\item [N2:] $\lVert a\rVert = 0 \Leftrightarrow a=0$
\item [N3:] $\forall\lambda\in\mathbb{R}:\lVert\lambda a\rVert = |\lambda |\lVert a\rVert$
\item [N4:] $\lVert a+b\rVert\leq\lVert a\rVert +\lVert b\rVert$
\end{itemize}
\end{document}
